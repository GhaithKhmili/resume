% LaTeX resume using res.cls
\documentclass[margin]{res}
%\usepackage{helvetica} % uses helvetica postscript font (download helvetica.sty)
%\usepackage{newcent}   % uses new century schoolbook postscript font 
\usepackage{xcolor}
\usepackage{setspace}
\usepackage[normalem]{ulem}
\usepackage{hyperref}
\hypersetup{
  colorlinks=true,
  urlcolor=blue
}
\setlength{\textwidth}{5.1in} % set width of text portion

\begin{document}

% Center the name over the entire width of resume:
 \moveleft.5\hoffset\centerline{\large\bf Georgios ``George'' Balatsouras}
% Draw a horizontal line the whole width of resume:
 \moveleft\hoffset\vbox{\hrule width\resumewidth height 1pt}\smallskip
% address begins here
% Again, the address lines must be centered over entire width of resume:
 \moveleft.5\hoffset\centerline{23 Achaion Street}
 \moveleft.5\hoffset\centerline{Agia Paraskevi (Athens 15343), Attica, Greece}
 \moveleft.5\hoffset\centerline{(Home) +30 210 6004683   (Cell) +30 6944 205788}
 \moveleft.5\hoffset\centerline{\href{mailto:gbalats@gmail.com}{\nolinkurl{gbalats@gmail.com}}}


\begin{resume}

\section{OBJECTIVE}
        To increase my knowledge and understand the practical issues of deductive databases
        by working for an innovative U.S. company that specializes in this field.

\section{EDUCATION}
        {\sl Bachelor of Science,} Computer Science \hfill 2004 - 2009  \\
        Ethnikon kai Kapodistriakon Panepistimion Athinon / University of Athens \\
        Department of Informatics \& Telecommunications \\
        Grade: 8.80 / 10

        {\sl Master of Science,} Computing Systems \hfill 2009 - 2012 \\
        Ethnikon kai Kapodistriakon Panepistimion Athinon / University of Athens \\
        Department of Informatics \& Telecommunications \\
        Grade: 9.78 / 10

\section{LANGUAGES} Greek (native), English (fluent)

\section{CERTIFICATES}
        {\sl Certificate of Proficiency in English, University of Michigan (USA)} \hfill 2003

\section{EXPERIENCE}

        {\sl Software Engineer Intern at LogicBlox} \hfill April 2012 - August 2012 \\
        Compiler Technologies at LogicBlox, Inc - Greater Atlanta Area
        \begin{itemize}
        \item Contributed to the active development of \texttt{BloxAnalysis},
          a tool used to perform static analysis of DatalogLB programs
          by representing them in a LogicBlox workspace
        \item Implemented a clone-detection library using \texttt{BloxAnalysis}
          that is able to: 
          \begin{itemize} 
          \item trace isomorphic formulas
          \item identify redundant rules in Datalog programs
          \item detect alternate indices that can be used in place of original predicates 
            for optimization purposes
          \end{itemize}
        \item Implemented a string-concatenation library that reduces the space complexity
          of the intermediate materialized strings from $O(n^2)$ to $O(nlogn)$
        \item Supervisor: Shan Shan Huang
          (\href{mailto:ssh@logicblox.com}{\nolinkurl{ssh@logicblox.com}})
        \end{itemize}

        {\sl Research Assistant} \hfill July 2011 - present \\
        University of Athens - Dept. of Informatics \& Telecommunications
        \begin{itemize} \itemsep -2pt %reduce space between items
        \item Project: PADECL (``Advanced Program Analysis Using Declarative Languages'')
        \item Supervisor: Assoc. Prof. Yannis Smaragdakis
          (\href{mailto:smaragd@di.uoa.gr}{\nolinkurl{smaragd@di.uoa.gr}})
        \end{itemize}
        
        {\sl Research Programmer} \hfill October 2010 - June 2011 \\
        University of Athens - Dept. of Informatics \& Telecommunications \\
        Project \href{http://pernasvip.di.uoa.gr/index.php}{PERNASVIP}
        \begin{itemize}
        \item Description: Development of the Itinerary Planner Server, i.e. the server who
          generates the 
          \href{http://pernasvip.di.uoa.gr/index.php/gen-spef/itinerary-example}{itineraries} 
          that will be used for navigation by the end-users of the system
        \item Supervisor: Prof. Alex Delis
          (\href{mailto:ad@di.uoa.gr}{\nolinkurl{ad@di.uoa.gr}})
        \end{itemize}

\section{ACTIVITIES}
        {\sl Teaching Assistant} \hfill 2009 - 2011 \\
        University of Athens - Dept. of Informatics \& Telecommunications

\section{TECHNICAL \\ SKILLS}
        {\sl Programming / Scripting Languages:} 
        C/C++, Java, Python, Prolog, Haskell, PHP, Bourne / Bash, Datalog, SQL

        {\sl Operating Systems:}
        Linux, Unix and Unix-like operating systems, BSD, Windows

        {\sl Database Management Systems and Tools:}
        MySQL, DBDesigner

        {\sl Compilers, Debugging Tools, Editors, IDEs, and Version Control Systems:}
        \begin{itemize}
        \item GCC/ G++, GDB, Valgrind
        \item Emacs, Eclipse, Geany
        \item Mercurial Version Control System
        \item Git Version Control System
        \item Microsoft Visual Studio 6, Visual Studio 2005
        \end{itemize}

        {\sl Other useful Platforms and Tools:}
        \begin{itemize}
        \item AspectJ: aspect-oriented extension for the Java Programming Language
        \item GNU Bison, Flex Lexical Analyzer
        \item ECLiPSe Constraint Programming System
        \item Xilinx ISE WebPACK
        \end{itemize}

        {\sl Bytecode Engineering Libraries:}
        ASM, BCEL

        {\sl Libraries and Standards:}
        CGAL, GNU MP Library, MPI, OpenGL, Pthreads

        {\sl Simulators:}
        SPIM, Xilinx ModelSim, SimpleScalar Simulator

        {\sl Other Skills:}
        Concurrent Programming, UNIX Systems Programming, Secure Coding in C, HTML + CSS, \LaTeX

\section{FIELDS OF INTEREST} Compilers, Programming Languages, Program Analysis, Logic Programming, \\ 
Deductive Databases, Operating Systems, System programming, Web Crawlers

\section{EXTRA-CURRICULAR \\ ACTIVITIES}             
Reading, Music (awarded a {\it Piano Certificate}), Soccer

\end{resume}
\end{document}




